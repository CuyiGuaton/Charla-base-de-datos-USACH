\documentclass[ignorenonframetext,]{beamer}
\setbeamertemplate{caption}[numbered]
\setbeamertemplate{caption label separator}{: }
\setbeamercolor{caption name}{fg=normal text.fg}
\usepackage{lmodern}
\usepackage{amssymb,amsmath}
\usepackage{ifxetex,ifluatex}
\usepackage{fixltx2e} % provides \textsubscript
\ifnum 0\ifxetex 1\fi\ifluatex 1\fi=0 % if pdftex
  \usepackage[T1]{fontenc}
  \usepackage[utf8]{inputenc}
\else % if luatex or xelatex
  \ifxetex
    \usepackage{mathspec}
  \else
    \usepackage{fontspec}
  \fi
  \defaultfontfeatures{Ligatures=TeX,Scale=MatchLowercase}
  \newcommand{\euro}{€}
\fi
% use upquote if available, for straight quotes in verbatim environments
\IfFileExists{upquote.sty}{\usepackage{upquote}}{}
% use microtype if available
\IfFileExists{microtype.sty}{%
\usepackage{microtype}
\UseMicrotypeSet[protrusion]{basicmath} % disable protrusion for tt fonts
}{}
\newif\ifbibliography
\usepackage{color}
\usepackage{fancyvrb}
\newcommand{\VerbBar}{|}
\newcommand{\VERB}{\Verb[commandchars=\\\{\}]}
\DefineVerbatimEnvironment{Highlighting}{Verbatim}{commandchars=\\\{\}}
% Add ',fontsize=\small' for more characters per line
\usepackage{framed}
\definecolor{shadecolor}{RGB}{248,248,248}
\newenvironment{Shaded}{\begin{snugshade}}{\end{snugshade}}
\newcommand{\KeywordTok}[1]{\textcolor[rgb]{0.13,0.29,0.53}{\textbf{{#1}}}}
\newcommand{\DataTypeTok}[1]{\textcolor[rgb]{0.13,0.29,0.53}{{#1}}}
\newcommand{\DecValTok}[1]{\textcolor[rgb]{0.00,0.00,0.81}{{#1}}}
\newcommand{\BaseNTok}[1]{\textcolor[rgb]{0.00,0.00,0.81}{{#1}}}
\newcommand{\FloatTok}[1]{\textcolor[rgb]{0.00,0.00,0.81}{{#1}}}
\newcommand{\ConstantTok}[1]{\textcolor[rgb]{0.00,0.00,0.00}{{#1}}}
\newcommand{\CharTok}[1]{\textcolor[rgb]{0.31,0.60,0.02}{{#1}}}
\newcommand{\SpecialCharTok}[1]{\textcolor[rgb]{0.00,0.00,0.00}{{#1}}}
\newcommand{\StringTok}[1]{\textcolor[rgb]{0.31,0.60,0.02}{{#1}}}
\newcommand{\VerbatimStringTok}[1]{\textcolor[rgb]{0.31,0.60,0.02}{{#1}}}
\newcommand{\SpecialStringTok}[1]{\textcolor[rgb]{0.31,0.60,0.02}{{#1}}}
\newcommand{\ImportTok}[1]{{#1}}
\newcommand{\CommentTok}[1]{\textcolor[rgb]{0.56,0.35,0.01}{\textit{{#1}}}}
\newcommand{\DocumentationTok}[1]{\textcolor[rgb]{0.56,0.35,0.01}{\textbf{\textit{{#1}}}}}
\newcommand{\AnnotationTok}[1]{\textcolor[rgb]{0.56,0.35,0.01}{\textbf{\textit{{#1}}}}}
\newcommand{\CommentVarTok}[1]{\textcolor[rgb]{0.56,0.35,0.01}{\textbf{\textit{{#1}}}}}
\newcommand{\OtherTok}[1]{\textcolor[rgb]{0.56,0.35,0.01}{{#1}}}
\newcommand{\FunctionTok}[1]{\textcolor[rgb]{0.00,0.00,0.00}{{#1}}}
\newcommand{\VariableTok}[1]{\textcolor[rgb]{0.00,0.00,0.00}{{#1}}}
\newcommand{\ControlFlowTok}[1]{\textcolor[rgb]{0.13,0.29,0.53}{\textbf{{#1}}}}
\newcommand{\OperatorTok}[1]{\textcolor[rgb]{0.81,0.36,0.00}{\textbf{{#1}}}}
\newcommand{\BuiltInTok}[1]{{#1}}
\newcommand{\ExtensionTok}[1]{{#1}}
\newcommand{\PreprocessorTok}[1]{\textcolor[rgb]{0.56,0.35,0.01}{\textit{{#1}}}}
\newcommand{\AttributeTok}[1]{\textcolor[rgb]{0.77,0.63,0.00}{{#1}}}
\newcommand{\RegionMarkerTok}[1]{{#1}}
\newcommand{\InformationTok}[1]{\textcolor[rgb]{0.56,0.35,0.01}{\textbf{\textit{{#1}}}}}
\newcommand{\WarningTok}[1]{\textcolor[rgb]{0.56,0.35,0.01}{\textbf{\textit{{#1}}}}}
\newcommand{\AlertTok}[1]{\textcolor[rgb]{0.94,0.16,0.16}{{#1}}}
\newcommand{\ErrorTok}[1]{\textcolor[rgb]{0.64,0.00,0.00}{\textbf{{#1}}}}
\newcommand{\NormalTok}[1]{{#1}}
\usepackage{longtable,booktabs}
\usepackage{caption}
% These lines are needed to make table captions work with longtable:
\makeatletter
\def\fnum@table{\tablename~\thetable}
\makeatother
\usepackage{graphicx,grffile}
\makeatletter
\def\maxwidth{\ifdim\Gin@nat@width>\linewidth\linewidth\else\Gin@nat@width\fi}
\def\maxheight{\ifdim\Gin@nat@height>\textheight0.8\textheight\else\Gin@nat@height\fi}
\makeatother
% Scale images if necessary, so that they will not overflow the page
% margins by default, and it is still possible to overwrite the defaults
% using explicit options in \includegraphics[width, height, ...]{}
\setkeys{Gin}{width=\maxwidth,height=\maxheight,keepaspectratio}

% Prevent slide breaks in the middle of a paragraph:
\widowpenalties 1 10000
\raggedbottom

% Comment these out if you don't want a slide with just the
% part/section/subsection/subsubsection title:
\AtBeginPart{
  \let\insertpartnumber\relax
  \let\partname\relax
  \frame{\partpage}
}
\AtBeginSection{
  \ifbibliography
  \else
    \let\insertsectionnumber\relax
    \let\sectionname\relax
    \frame{\sectionpage}
  \fi
}
\AtBeginSubsection{
  \let\insertsubsectionnumber\relax
  \let\subsectionname\relax
  \frame{\subsectionpage}
}

\usepackage[normalem]{ulem}
% avoid problems with \sout in headers with hyperref:
\pdfstringdefDisableCommands{\renewcommand{\sout}{}}
\setlength{\emergencystretch}{3em}  % prevent overfull lines
\providecommand{\tightlist}{%
  \setlength{\itemsep}{0pt}\setlength{\parskip}{0pt}}
\setcounter{secnumdepth}{0}

\title{Habits}
\author{John Doe}
\date{March 22, 2005}

\begin{document}
\frame{\titlepage}

\begin{frame}{Base de datos para tu proyecto}

\begin{itemize}
\tightlist
\item
  ¿Qué es una base de datos?
\item
  BD u hojas de calculo
\item
  Implementación

  \begin{enumerate}
  \def\labelenumi{\arabic{enumi}.}
  \tightlist
  \item
    Pico
  \end{enumerate}
\end{itemize}

\end{frame}

\begin{frame}{==Base de dato== vs hoja de calculo}

\begin{longtable}[c]{@{}lllllll@{}}
\toprule
\begin{minipage}[b]{0.17\columnwidth}\raggedright\strut
Nombre Paciente
\strut\end{minipage} &
\begin{minipage}[b]{0.06\columnwidth}\raggedright\strut
Tipo
\strut\end{minipage} &
\begin{minipage}[b]{0.08\columnwidth}\raggedright\strut
Síntomas
\strut\end{minipage} &
\begin{minipage}[b]{0.07\columnwidth}\raggedright\strut
Medico
\strut\end{minipage} &
\begin{minipage}[b]{0.06\columnwidth}\raggedright\strut
Rut
\strut\end{minipage} &
\begin{minipage}[b]{0.08\columnwidth}\raggedright\strut
E.C.
\strut\end{minipage} &
\begin{minipage}[b]{0.10\columnwidth}\raggedright\strut
Sueldo
\strut\end{minipage}\tabularnewline
\midrule
\endhead
\begin{minipage}[t]{0.17\columnwidth}\raggedright\strut
Sasha
\strut\end{minipage} &
\begin{minipage}[t]{0.06\columnwidth}\raggedright\strut
Felino
\strut\end{minipage} &
\begin{minipage}[t]{0.08\columnwidth}\raggedright\strut
Vomito, cansancio, pelo caído
\strut\end{minipage} &
\begin{minipage}[t]{0.07\columnwidth}\raggedright\strut
Álvaro Pérez
\strut\end{minipage} &
\begin{minipage}[t]{0.06\columnwidth}\raggedright\strut
16.336.789-7
\strut\end{minipage} &
\begin{minipage}[t]{0.08\columnwidth}\raggedright\strut
Soltero
\strut\end{minipage} &
\begin{minipage}[t]{0.10\columnwidth}\raggedright\strut
\$500.000
\strut\end{minipage}\tabularnewline
\begin{minipage}[t]{0.17\columnwidth}\raggedright\strut
Luna
\strut\end{minipage} &
\begin{minipage}[t]{0.06\columnwidth}\raggedright\strut
Felino
\strut\end{minipage} &
\begin{minipage}[t]{0.08\columnwidth}\raggedright\strut
Un poco vaga
\strut\end{minipage} &
\begin{minipage}[t]{0.07\columnwidth}\raggedright\strut
Álvaro Pérez
\strut\end{minipage} &
\begin{minipage}[t]{0.06\columnwidth}\raggedright\strut
16.336.789-7
\strut\end{minipage} &
\begin{minipage}[t]{0.08\columnwidth}\raggedright\strut
Soltero
\strut\end{minipage} &
\begin{minipage}[t]{0.10\columnwidth}\raggedright\strut
\$500.000
\strut\end{minipage}\tabularnewline
\begin{minipage}[t]{0.17\columnwidth}\raggedright\strut
Toby
\strut\end{minipage} &
\begin{minipage}[t]{0.06\columnwidth}\raggedright\strut
Canino
\strut\end{minipage} &
\begin{minipage}[t]{0.08\columnwidth}\raggedright\strut
No come
\strut\end{minipage} &
\begin{minipage}[t]{0.07\columnwidth}\raggedright\strut
Juan Piedra
\strut\end{minipage} &
\begin{minipage}[t]{0.06\columnwidth}\raggedright\strut
15.533.559-5
\strut\end{minipage} &
\begin{minipage}[t]{0.08\columnwidth}\raggedright\strut
Soltero
\strut\end{minipage} &
\begin{minipage}[t]{0.10\columnwidth}\raggedright\strut
\$700.000
\strut\end{minipage}\tabularnewline
\bottomrule
\end{longtable}

\end{frame}

\begin{frame}

{[}\url{https://www.gcfaprendelibre.org/tecnologia/curso/access_2010/trabajar_con_bases_de_datos/1.do}{]}

\end{frame}

\begin{frame}{Marp}

\end{frame}

\begin{frame}{\includegraphics{images/marp.png}}

\begin{block}{Markdown presentation writer, powered by
\href{http://electron.atom.io/}{Electron}}

\begin{block}{Created by Yuki Hattori (
{[}@yhatt{]}(\url{https://github.com/yhatt}) )}

\end{block}

\end{block}

\end{frame}

\begin{frame}{Features}

\begin{itemize}
\tightlist
\item
  \textbf{Slides are written in Markdown.}
\item
  Cross-platform. Supports Windows, Mac, and Linux
\item
  Live Preview with 3 modes
\item
  Slide themes (\texttt{default}, \texttt{gaia}) and custom background
  images
\item
  Supports emoji :heart:
\item
  Render maths in your slides
\item
  Export your slides to PDF
\end{itemize}

\end{frame}

\begin{frame}[fragile]{How to write slides?}

Split slides by horizontal ruler \texttt{-\/-\/-}. It's very simple.

\begin{verbatim}
# Slide 1

foobar

---

# Slide 2

foobar
\end{verbatim}

\begin{quote}
\emph{Notice: Ruler (\texttt{\textless{}hr\textgreater{}}) is not
displayed in Marp.}
\end{quote}

\end{frame}

\begin{frame}[fragile]{Directives}

Marp's Markdown has extended directives to affect slides.

Insert HTML comment as below:

\begin{Shaded}
\begin{Highlighting}[]
\CommentTok{<!-- \{directive_name\}: \{value\} -->}
\end{Highlighting}
\end{Shaded}

\begin{Shaded}
\begin{Highlighting}[]
\CommentTok{<!--}
\CommentTok{\{first_directive_name\}:  \{value\}}
\CommentTok{\{second_directive_name\}: \{value\}}
\CommentTok{...}
\CommentTok{-->}
\end{Highlighting}
\end{Shaded}

\end{frame}

\begin{frame}[fragile]

\begin{block}{Global Directives}

\begin{block}{\texttt{\$theme}}

Changes the theme of all the slides in the deck. You can also change
from \texttt{View\ -\textgreater{}\ Theme} menu.

\begin{verbatim}
<!-- $theme: gaia -->
\end{verbatim}

\begin{longtable}[c]{@{}ccl@{}}
\toprule
Theme name & Value & Directive\tabularnewline
\midrule
\endhead
\textbf{\emph{Default}} & default &
\texttt{\textless{}!-\/-\ \$theme:\ default\ -\/-\textgreater{}}\tabularnewline
\textbf{Gaia} & gaia &
\texttt{\textless{}!-\/-\ \$theme:\ gaia\ -\/-\textgreater{}}\tabularnewline
\bottomrule
\end{longtable}

\end{block}

\end{block}

\end{frame}

\begin{frame}[fragile]

\begin{block}{\texttt{\$width} / \texttt{\$height}}

Changes width and height of all the slides.

You can use units: \texttt{px} (default), \texttt{cm}, \texttt{mm},
\texttt{in}, \texttt{pt}, and \texttt{pc}.

\begin{Shaded}
\begin{Highlighting}[]
\CommentTok{<!-- $width: 12in -->}
\end{Highlighting}
\end{Shaded}

\end{block}

\begin{block}{\texttt{\$size}}

Changes slide size by presets.

Presets: \texttt{4:3}, \texttt{16:9}, \texttt{A0}-\texttt{A8},
\texttt{B0}-\texttt{B8} and suffix of \texttt{-portrait}.

\begin{Shaded}
\begin{Highlighting}[]
\CommentTok{<!-- $size: 16:9 -->}
\end{Highlighting}
\end{Shaded}

\end{block}

\end{frame}

\begin{frame}[fragile]

\begin{block}{Page Directives}

The page directive would apply to the \textbf{current page and the
following pages}. You should insert it \emph{at the top} to apply it to
all slides.

\begin{block}{\texttt{page\_number}}

Set \texttt{true} to show page number on slides. \emph{See lower right!}

\begin{Shaded}
\begin{Highlighting}[]
\CommentTok{<!-- page_number: true -->}
\end{Highlighting}
\end{Shaded}

\end{block}

\end{block}

\end{frame}

\begin{frame}[fragile]

\begin{block}{\texttt{template}}

Set to use template of theme.

The \texttt{template} directive just enables that using theme supports
templates.

\begin{Shaded}
\begin{Highlighting}[]
\CommentTok{<!--}
\CommentTok{$theme: gaia}
\CommentTok{template: invert}
\CommentTok{-->}

\NormalTok{Example: Set "invert" template of Gaia theme.}
\end{Highlighting}
\end{Shaded}

\end{block}

\end{frame}

\begin{frame}[fragile]

\begin{block}{\texttt{footer}}

Add a footer to the current slide and all of the following slides

\begin{Shaded}
\begin{Highlighting}[]
\CommentTok{<!-- footer: This is a footer -->}
\end{Highlighting}
\end{Shaded}

Example: Adds ``This is a footer'' in the bottom of each slide

\end{block}

\end{frame}

\begin{frame}[fragile]

\begin{block}{\texttt{prerender}}

Pre-renders a slide, which can prevent issues with very large background
images.

\begin{Shaded}
\begin{Highlighting}[]
\CommentTok{<!-- prerender: true -->}
\end{Highlighting}
\end{Shaded}

\end{block}

\end{frame}

\begin{frame}[fragile]

\begin{block}{Pro Tips}

\begin{block}{Apply page directive to current slide only}

Page directive can be selectively applied to the current slide by
prefixing the page directive with \texttt{*}.

\begin{verbatim}
<!-- *page_number: false -->
<!-- *template: invert -->
\end{verbatim}

\end{block}

\end{block}

\end{frame}

\begin{frame}[fragile]

\begin{block}{Slide background Images}

You can set an image as a slide background.

\begin{Shaded}
\begin{Highlighting}[]
\NormalTok{![bg](mybackground.png)}
\end{Highlighting}
\end{Shaded}

Options can be provided after \texttt{bg}, for example
\texttt{!{[}bg\ original{]}(path)}.

Options include:

\begin{itemize}
\tightlist
\item
  \texttt{original} to include the image without any effects
\item
  \texttt{x\%} to include the image at \texttt{x} percent of the slide
  size
\end{itemize}

Include multiple\texttt{!{[}bg{]}(path)} tags to stack background images
horizontally.

\begin{figure}[htbp]
\centering
\includegraphics{images/background.png}
\caption{bg}
\end{figure}

\end{block}

\end{frame}

\begin{frame}[fragile]

\begin{block}{Maths Typsetting}

Mathematics is typeset using the \texttt{KaTeX} package. Use \texttt{\$}
for inline maths, such as \(ax^2+bc+c\), and \texttt{\$\$} for block
maths:

\[I_{xx}=\int\int_Ry^2f(x,y)\cdot{}dydx\]

\begin{Shaded}
\begin{Highlighting}[]
\NormalTok{This is inline: $ax^2+bx+c$, and this is block:}

\NormalTok{$$I_\{xx\}=\textbackslash{}int\textbackslash{}int_Ry^2f(x,y)\textbackslash{}cdot\{\}dydx$$}
\end{Highlighting}
\end{Shaded}

\end{block}

\end{frame}

\begin{frame}

\begin{block}{Enjoy writing slides! :+1:}

\begin{block}{\url{https://github.com/yhatt/marp}}

Copyright © 2016 \href{https://github.com/yhatt}{Yuki Hattori} This
software released under the
\href{https://github.com/yhatt/marp/blob/master/LICENSE}{MIT License}.

\end{block}

\end{block}

\end{frame}

\begin{frame}{Introducing ==Gaia== theme}

\begin{block}{Marp's new slide theme}

\begin{block}{Created by {[}Yuki Hattori
(@yhatt){]}(\url{https://github.com/yhatt})}

\end{block}

\end{block}

\end{frame}

\begin{frame}{Overview}

\textbf{Gaia} is the beautiful presentation theme on Marp!

\begin{itemize}
\tightlist
\item
  ==\textbf{New features}==

  \begin{enumerate}
  \def\labelenumi{\arabic{enumi}.}
  \tightlist
  \item
    Title Slides
  \item
    Highlight
  \item
    Color scheme
  \end{enumerate}
\end{itemize}

\end{frame}

\begin{frame}[fragile]{How to use}

\begin{block}{From menu}

Select menu: \emph{View :arrow\_right: Theme :arrow\_right: Gaia}

\end{block}

\begin{block}{Use directive}

Set \texttt{gaia} theme by \texttt{\$theme} Global Directive.

\begin{verbatim}
<!-- $theme: gaia -->
\end{verbatim}

\end{block}

\end{frame}

\begin{frame}{Basic example 1}

\textbf{Lorem ipsum} dolor \emph{sit} amet, \textbf{\emph{consectetur}}
adipiscing elit, sed do \texttt{eiusmod} tempor ==incididunt== ut labore
et dolore \sout{magna aliqua}. :smile:

\begin{quote}
Stay Hungry. Stay Foolish. \emph{--Steve Jobs (2005)}
\end{quote}

\begin{itemize}
\tightlist
\item
  List A

  \begin{enumerate}
  \def\labelenumi{\arabic{enumi}.}
  \tightlist
  \item
    \href{https://yhatt.github.io/marp/}{Sub list}
  \item
    Sub list

    \begin{itemize}
    \tightlist
    \item
      \emph{More Sub list}
    \end{itemize}
  \end{enumerate}
\end{itemize}

\end{frame}

\begin{frame}[fragile]{Basic example 2}

\begin{Shaded}
\begin{Highlighting}[]
\VariableTok{document}\NormalTok{.}\AttributeTok{write}\NormalTok{(}\StringTok{'Hello, world!'}\NormalTok{)}\OperatorTok{;}
\end{Highlighting}
\end{Shaded}

\begin{longtable}[c]{@{}lcr@{}}
\toprule
\begin{minipage}[b]{0.05\columnwidth}\raggedright\strut
table
\strut\end{minipage} &
\begin{minipage}[b]{0.05\columnwidth}\centering\strut
layout
\strut\end{minipage} &
\begin{minipage}[b]{0.05\columnwidth}\raggedleft\strut
example
\strut\end{minipage}\tabularnewline
\midrule
\endhead
\begin{minipage}[t]{0.05\columnwidth}\raggedright\strut
align to left
\strut\end{minipage} &
\begin{minipage}[t]{0.05\columnwidth}\centering\strut
align to center
\strut\end{minipage} &
\begin{minipage}[t]{0.05\columnwidth}\raggedleft\strut
align to right
\strut\end{minipage}\tabularnewline
\begin{minipage}[t]{0.05\columnwidth}\raggedright\strut
:arrow\_left: left
\strut\end{minipage} &
\begin{minipage}[t]{0.05\columnwidth}\centering\strut
:arrow\_left: center :arrow\_right:
\strut\end{minipage} &
\begin{minipage}[t]{0.05\columnwidth}\raggedleft\strut
right :arrow\_right:
\strut\end{minipage}\tabularnewline
\bottomrule
\end{longtable}

\begin{figure}[htbp]
\centering
\includegraphics{../images/marp.png}
\caption{70\% center}
\end{figure}

\end{frame}

\begin{frame}{==e.g.== This page :yum:}

\end{frame}

\begin{frame}

\begin{block}{==Apply centering== to the pagethat has only headings!}

\begin{block}{Useful to title slide. :laughing:}

\end{block}

\end{block}

\end{frame}

\begin{frame}[fragile]

\begin{quote}
\textbf{==Tips:==} Apply vertical centering to quote only page too.
\end{quote}

\begin{block}{Highlight Markup}

You can use \texttt{==} for ==highlighting blue==.

\begin{Shaded}
\begin{Highlighting}[]
\NormalTok{==This is highlight markup.==}
\end{Highlighting}
\end{Shaded}

\begin{block}{Notice}

\emph{Marp would show yellow marker highlight in Markdown view or
default slide theme.}

\end{block}

\end{block}

\end{frame}

\begin{frame}[fragile]{==Color== scheme templates}

Change color scheme \emph{by \texttt{template} page directive.}

\begin{verbatim}
<!-- template: default -->
\end{verbatim}

\begin{itemize}
\tightlist
\item
  \textbf{Default} :arrow\_left: This page
\item
  Invert
\item
  Gaia (Theme color)
\end{itemize}

\end{frame}

\begin{frame}[fragile]{==Color== scheme templates}

Change color scheme \emph{by \texttt{template} page directive.}

\begin{verbatim}
<!-- template: gaia -->
\end{verbatim}

\begin{itemize}
\tightlist
\item
  Default
\item
  Invert
\item
  \textbf{Gaia} (Theme color) :arrow\_left: This page
\end{itemize}

\end{frame}

\begin{frame}{==That's all!==}

\begin{block}{Let's create beautiful slideswith ==Marp== + ==Gaia==
theme!}

\end{block}

\end{frame}

\begin{frame}

\begin{block}{\texttt{\textless{}!-\/-\ \$theme:\ gaia\ -\/-\textgreater{}}
of Marp}

\begin{block}{\href{https://yhatt.github.io/marp}{\includegraphics{../images/marp.png}}}

\end{block}

\end{block}

\begin{block}{\url{https://yhatt.github.io/marp}}

\end{block}

\end{frame}

\end{document}
